% !TEX TS-program = pdflatex
% !TEX encoding = UTF-8 Unicode

% This is a simple template for a LaTeX document using the "article" class.
% See "book", "report", "letter" for other types of document.

\documentclass[11pt]{article} % use larger type; default would be 10pt

\usepackage[utf8]{inputenc} % set input encoding (not needed with XeLaTeX)

%%% Examples of Article customizations
% These packages are optional, depending whether you want the features they provide.
% See the LaTeX Companion or other references for full information.

%%% PAGE DIMENSIONS
\usepackage{geometry} % to change the page dimensions
\geometry{a4paper} % or letterpaper (US) or a5paper or....
% \geometry{margins=2in} % for example, change the margins to 2 inches all round
% \geometry{landscape} % set up the page for landscape
%   read geometry.pdf for detailed page layout information

\usepackage{graphicx} % support the \includegravarphics command and options

% \usepackage[parfill]{parskip} % Activate to begin paragraphs with an empty line rather than an indent

%%% PACKAGES
\usepackage{booktabs} % for much better looking tables
\usepackage{array} % for better arrays (eg matrices) in maths
\usepackage{paralist} % very flexible & customisable lists (eg. enumerate/itemize, etc.)
\usepackage{verbatim} % adds environment for commenting out blocks of text & for better verbatim
\usepackage{subfig} % make it possible to include more than one captioned figure/table in a single float
% These packages are all incorporated in the memoir class to one degree or another...

\usepackage{amsmath} 


%%% HEADERS & FOOTERS
\usepackage{fancyhdr} % This should be set AFTER setting up the page geometry
\pagestyle{fancy} % options: empty , plain , fancy
\renewcommand{\headrulewidth}{0pt} % customise the layout...
\lhead{}\chead{}\rhead{}
\lfoot{}\cfoot{\thepage}\rfoot{}

%%% SECTION TITLE APPEARANCE
\usepackage{sectsty}
\allsectionsfont{\sffamily\mdseries\upshape} % (See the fntguide.pdf for font help)
% (This matches ConTeXt defaults)

%%% ToC (table of contents) APPEARANCE
\usepackage[nottoc,notlof,notlot]{tocbibind} % Put the bibliography in the ToC
\usepackage[titles,subfigure]{tocloft} % Alter the style of the Table of Contents
\renewcommand{\cftsecfont}{\rmfamily\mdseries\upshape}
\renewcommand{\cftsecpagefont}{\rmfamily\mdseries\upshape} % No bold!

%%% END Article customizations

%%% The "real" document content comes below...

\title{Subset of Formulas for Ray Tracing }
\author{Sasan Ardalan, https://www.radiocalc.com }
%\date{} % Activate to display a given date or no date (if empty),
         % otherwise the current date is printed 

\begin{document}
\maketitle

\section{Introduction}

These are key sets of formula for ray tracing from Jones and Stephenson (1975)

                                          
\subsection{Reference}

R. Michael Jones, Judith J. 
  Stephenson,"A Versatile Three-Dimensional  Ray Tracing 
Compute Program for Radio Waves in the Ionosphere,"             
 OT Report 75-75, US Department of Commerce,  October 1975       

\section{Formulas}

\equation
H(r,\theta,\varphi, k_r,k_\theta, k_\varphi) = \frac{1}{2}Re \left [\frac{c^2}{\omega^2}(k_r^2+k_\theta^2+k_\varphi^2)-n^2 \right]
\endequation


\equation
\frac{dr}{d\tau}=\frac{\partial H}{\partial k_r}
\endequation

\equation
\frac{d\theta}{d\tau}=\frac{1}{r}\frac{\partial H}{\partial k_\theta}
\endequation


\equation
\frac{d\varphi}{d\tau}=\frac{1}{r\sin\theta}\frac{\partial H}{\partial k_\varphi}
\endequation

\equation
\frac{dk_r}{d\tau}=-\frac{\partial H}{\partial r}+k_\theta\frac{d\theta}{d\tau}+k_\varphi\sin\theta\frac{d\varphi}{d\tau}
\endequation


\equation
\frac{dk_\theta}{d\tau}=\frac{1}{r}(-\frac{\partial H}{\partial\theta}-k_\theta\frac{dr}{d\tau}+k_\varphi r \cos \theta\frac{d\varphi}{d\tau})
\endequation


\equation
 \frac{dk_\varphi}{d\tau}=\frac{1}{r\sin\theta }( -\frac{\partial H}{\partial \varphi}-k_\varphi\sin \theta \frac{dr}{d\tau}-k_\varphi r \cos \theta \frac{d\theta}{d\tau})
\endequation






\equation
\frac{d\omega}{d\tau}=\frac{\partial H}{\partial t}
\endequation

 \equation
 R(1)=r
\endequation
\equation
 R(2)=\theta
\endequation
\equation
 R(3)=\varphi
\endequation
\equation
 R(4)=k_r
\endequation
\equation
 R(5)=k_\theta
\endequation
\equation
 R(6)=k_\varphi
\endequation


Note R(7) through R(10) Variables  the User Chooses to Integrate.

\equation
 R(7)=P \text{ Phase  Path in Kilometers}
\endequation

\equation
 R(8)=A \text{ Absorption in Decibels}
\endequation

\equation
 R(9)=\Delta f  \text{ Doppler Shift in Hertz}
\endequation

\equation
 R(10)=s \text{ Geometrical Path Length in Kilometers}
\endequation


 

\begin{table}[]
\caption{List of Symbols}
\label{my-label}
\begin{tabular}{ll}
 
      $\lambda$                                  & Wavelength                                                                                                                                                                                                                                 \\ 
      $\lambda_0$                                & Wavelength Free Space                                                                                                                                                                                                                      \\
 $\tau$                                          & Independent variable in Hamilton's Equations                                                                                                                                                                                               \\
$\varphi$                                        & Longitude in spherical polar coordinates                                                                                                                                                                                                   \\
$\omega$                                         & $2\pi f$, angular wave frequency                                                                                                                                                                               \\
$ \theta $                                       & Colatitude in spherical polar coordinates                                                                                                                                                                                                   \\
$ k_r, k_\theta, k_\varphi $                     & Components of the propagation vector in the r,$ \theta$ , $\varphi$ directions 
\\
  & -- a vector perpendicular to the wave front having a magnitude $\frac{2 \pi}{ \lambda}=\frac{\omega}{v} $
\\
f                                                & Wave frequency                                                                                                                                                                                                                             \\
n                                                & Phase refractive index (in general complex)                                                                                                                                                                                                \\
r,$\theta$ , $\varphi$                           & Coordinates of a point in spherical polar coordinates                                                                                                                                                                                       \\
s                                                & Geometric ray path length                                                                                                                                                                                                                  \\
c                                                & Speed of electromagnetic waves in free space.                                                                                                                                                                                              \\
t                                                & Time, travel time of a wave packet.                                                                                                                                                                                                        \\
$\epsilon_0$                                    & Electric permittivity of free space                                                                                                                                                                                                        \\
                                                 &                                                                                                                                                                                                                                           
\end{tabular}
\end{table}




\end{document}
